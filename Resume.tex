  %-------------------------
  % Resume in Latex
  % Author : Jake Gutierrez
  % Based off of: https://github.com/sb2nov/resume
  % License : MIT
  %------------------------

  \documentclass[letterpaper,11pt]{article}

  \usepackage{latexsym}
  \usepackage[empty]{fullpage}
  \usepackage{titlesec}
  \usepackage{marvosym}
  \usepackage[usenames,dvipsnames]{color}
  \usepackage{verbatim}
  \usepackage{enumitem}
  \usepackage[hidelinks]{hyperref}
  \usepackage{fancyhdr}
  \usepackage[english]{babel}
  \usepackage{tabularx}
  \input{glyphtounicode}
  \usepackage{multicol}


  %----------FONT OPTIONS----------
  % sans-serif
  % \usepackage[sfdefault]{FiraSans}
  % \usepackage[sfdefault]{roboto}
  % \usepackage[sfdefault]{noto-sans}
  % \usepackage[default]{sourcesanspro}

  % serif
  % \usepackage{CormorantGaramond}
  % \usepackage{charter}


  \pagestyle{fancy}
  \fancyhf{} % clear all header and footer fields
  \fancyfoot{}
  \renewcommand{\headrulewidth}{0pt}
  \renewcommand{\footrulewidth}{0pt}

  % Adjust margins
  \addtolength{\oddsidemargin}{-0.5in}
  \addtolength{\evensidemargin}{-0.5in}
  \addtolength{\textwidth}{1in}
  \addtolength{\topmargin}{-.5in}
  \addtolength{\textheight}{1.0in}

  \urlstyle{same}

  \raggedbottom
  \raggedright
  \setlength{\tabcolsep}{0in}

  % Sections formatting
  \titleformat{\section}{
    \vspace{-4pt}\scshape\raggedright\large
  }{}{0em}{}[\color{black}\titlerule \vspace{-5pt}]

  % Ensure that generate pdf is machine readable/ATS parsable
  \pdfgentounicode=1

  %-------------------------
  % Custom commands
  \newcommand{\resumeItem}[1]{
    \item\small{
      {#1 \vspace{-2pt}}
    }
  }

  \newcommand{\resumeSubheading}[4]{
    \vspace{-2pt}\item
      \begin{tabular*}{0.97\textwidth}[t]{l@{\extracolsep{\fill}}r}
        \textbf{#1} & #2 \\
        \textit{\small#3} & \textit{\small #4} \\
      \end{tabular*}\vspace{-7pt}
  }

  \newcommand{\resumeSubSubheading}[2]{
      \item
      \begin{tabular*}{0.97\textwidth}{l@{\extracolsep{\fill}}r}
        \textit{\small#1} & \textit{\small #2} \\
      \end{tabular*}\vspace{-7pt}
  }

  \newcommand{\resumeProjectHeading}[2]{
      \item
      \begin{tabular*}{0.97\textwidth}{l@{\extracolsep{\fill}}r}
        \small#1 & #2 \\
      \end{tabular*}\vspace{-7pt}
  }

  \newcommand{\resumeSubItem}[1]{\resumeItem{#1}\vspace{-4pt}}

  \renewcommand\labelitemii{$\vcenter{\hbox{\tiny$\bullet$}}$}

  \newcommand{\resumeSubHeadingListStart}{\begin{itemize}[leftmargin=0.15in, label={}]}
  \newcommand{\resumeSubHeadingListEnd}{\end{itemize}}
  \newcommand{\resumeItemListStart}{\begin{itemize}}
  \newcommand{\resumeItemListEnd}{\end{itemize}\vspace{-5pt}}

  %-------------------------------------------
  %%%%%%  RESUME STARTS HERE  %%%%%%%%%%%%%%%%%%%%%%%%%%%%


\begin{document}

\begin{center}
  \textbf{\Huge \scshape Robert C. Wong} \\ \vspace{1pt}
  \small 917-993-4624 $|$ \href{mailto:robwon15@gmail.com}{\underline{robwong15@gmail.com}} $|$
  \href{https://linkedin.com/in/robertcwong}{\underline{linkedin.com/in/robertcwong}} $|$
  \href{https://github.com/robw0ng/}{\underline{github.com/robw0ng}}
  \\
  \href{https://robw0ng.github.io/NameCube/}{\underline{robw0ng.github.io/NameCube/}}

\end{center}

%-----------EDUCATION-----------
\section{Education}
\resumeSubHeadingListStart
\resumeSubheading
{Stony Brook University}{Stony Brook, NY}
{Bachelor of Science in Computer Science}{}
% Aug. 2022 -- May 2026
\begin{itemize}[leftmargin=0.15in, label={}, itemsep=10pt, topsep=5pt, parsep=0pt, partopsep=0pt]
  \item \small{\textbf{Relevant Coursework:} \textit{Software Development, Software Security, System Fundamentals 1 \& 2, Data Structures \& Algorithms, Analysis of Algorithms, Programming Abstractions, Discrete Mathematics, Linear Algebra}}
\end{itemize}
\resumeSubHeadingListEnd
\vspace{-12pt}

%-----------EXPERIENCE-----------
\section{Experience}
\resumeSubHeadingListStart

\resumeSubheading
{Software Developer - Summer College Intern}{Jul. 2024 - Aug. 2024}
{NYPD Compliance Division}{Manhattan, New York}
\resumeItemListStart
\resumeItem{Tackled issues concering big data processing, user submission forms, and database management bettering the quality of life for staff working on their own tasks.}
\resumeItem{Developed user-friendly applications to transform complex and inaccessible systems into easily navigable tools for staff with and without limited computer experience, ensuring accessibility and ease of use for all users.}
\resumeItem{Singlehandedly improved departmental efficiency by \textbf{86.35\%} through the development and deployment of custom software solutions.}
\resumeItem{Created and implemented programs that are now in constant use by numerous team members, streamlining their daily responsibilities and significantly enhancing operational effectiveness across the department.}
\resumeItemListEnd

\resumeSubheading
{AI Trainer}{Jun. 2024 - Present}
{Outlier AI}{Remote}
\resumeItemListStart
\resumeItem{Analyzed the output and backend of an AI model according to given prompts, reviewing the steps the model took, which included Python and SQL processes.}

% \resumeItem{Assessed which output was best and provided a detailed justification regarding it.}

\resumeItem{Contributed to the AI model's improvement by providing comprehensive feedback and justifications, enhancing its ability to generate accurate and relevant responses for future prompts.}

% \resumeItem{Maintained detailed records of evaluations and justifications to track progress and identify areas for further enhancement.}
\resumeItemListEnd

%           \resumeSubheading
%       {Indie Game Developer}{Jan 2016 - April 2020}
%       {Roblox}{Remote}
%       \resumeItemListStart
%         \resumeItem{Started and led Octagon Productions, a team of developers dedicated to creating various games to be sold or played by users on the platform.}

% \resumeItem{Utilized Lua and Roblox's 3D modeling tools to create assets and scripts for the various builds for our projects.}
%         \resumeItem{Created Precision, a first-person shooter. Amassing about 40 thousand unique visits with a small community of players that enjoyed the game.}

%         \resumeItem{Commissioned and sold builds to others for their own use. Creating games and specific builds for certain groups on the platform.}
%       \resumeItemListEnd

\resumeSubHeadingListEnd

%-----------PROJECTS-----------
\section{Projects}
\resumeSubHeadingListStart

\resumeProjectHeading
{\textbf{RECAP - Data Entry} $|$ \emph{Python, Flask, SQLite, JavaScript, HTML/CSS}}{Aug. 2024 -- Sept. 2024}
\resumeItemListStart
\resumeItem{Developed an internal application for the NYPD's Body-Worn Camera unit to streamline the documentation and management of video evidence related to force incidents.}
\resumeItem{Reduced form submission time by \textbf{58\%}, significantly improving efficiency and productivity.}
\resumeItem{Designed a user-friendly front-end interface using HTML, CSS, and the Bootstrap framework, enabling seamless form submissions and data retrieval by coworkers.}
\resumeItem{Developed a Flask-based backend with Python and SQLite for efficient data management, using JavaScript and AJAX to dynamically display database records on the website for management to view.}
\resumeItemListEnd

\resumeProjectHeading
{\textbf{ComplianceGenie - Data Processing} $|$ \emph{Python, pandas, python-pptx, CustomTkinter}}{Jun. 2024 -- Aug. 2024}
\resumeItemListStart
\resumeItem{A Python-based application for the NYPD to automate the generation of reports and presentations,
  improving efficiency by \textbf{98.5\%}, replacing manual computations and data entry.}
\resumeItem{Utilized pandas for data processing to read, clean, and compile incident, interaction, and arrest data.}
\resumeItem{Designed an intuitive front-end using CustomTkinter, prioritizing ease of use and utility for non-technical staff.}
\resumeItem{Generated both citywide and borough-specific statistics. Outputting results in comprehensive Excel reports using openpyxl and PowerPoint presentations using python-pptx.}

\resumeItemListEnd
\resumeProjectHeading
{\textbf{Legion - Daemon Manager} $|$ \emph{C, Unix System Calls, Signal Handling, Process Control}}{Mar. 2024 -- Apr. 2024}
\resumeItemListStart
% \resumeItem{Developed an application focusing on process handling and inter-process communication in a Unix environment}
% \resumeItem{Involved creating, and managing daemons, implementing functions such as starting, stopping, and logging daemon activities efficiently.}
% \resumeItem{Employed inter-process communication techniques such as pipes and signal handlers to synchronize and control daemon processes securely and effectively.}
% \resumeItem{Implemented error handling and signal safety mechanisms to ensure system stability and reliability during daemon runtime operations.}
\resumeItem{Developed a Unix application for process handling and inter-process communication. }
\resumeItem{Created and managed daemons with functions for starting, stopping, and logging activities.}
\resumeItem{Employed pipes and signal handlers for secure and effective synchronization and control.}
\resumeItem{Ensured system stability with robust error handling and signal safety mechanisms.}
\resumeItemListEnd



% \resumeProjectHeading
%     {\textbf{Charla - Chat Server} $|$ \emph{C, POSIX Threads, Sockets, CSAPP Library}}{April 2024 -- May 2024}
%     \resumeItemListStart
%       % \resumeItem{Developed a multi-threaded chat server in C that handles multiple client connections simultaneously, demonstrating proficiency in network programming and concurrent computing.}
%       % \resumeItem{Utilized POSIX sockets to manage client connections, enabling real-time messaging between users.}
%       % \resumeItem{Implemented thread-safe operations using mutexes and semaphores to manage access to shared resources among multiple threads, ensuring data integrity and preventing race conditions.}
%       % \resumeItem{Designed and implemented user and client registries with functionalities to register, unregister, and query clients and users, supporting dynamic user sessions.}
%       \resumeItem{Developed a multi-threaded chat server in C for handling multiple client connections. }
%       \resumeItem{Utilized POSIX sockets for real-time messaging between users. }
%       \resumeItem{Ensured thread-safe operations with mutexes and semaphores to manage shared resources. }
%       \resumeItem{Designed registries for dynamic user sessions with functionalities for registration and querying. }
%     \resumeItemListEnd

% \resumeProjectHeading
%           {\textbf{Chip-8} $|$ \emph{Rust, SDL, Graphics and Input Handling }}{June 2024 -- July 2024}
%           \resumeItemListStart
%             \resumeItem{Developed a fully functional Chip-8 emulator capable of running classic Chip-8 games and programs.}
%             \resumeItem{Emulated key components including RAM, program counter, a variable register, instruction register, stack pointer, stack, keypad for input, delay timer, and sound timer.}
%             \resumeItem{Implemented the emulator using Rust and utilized SDL for rendering the 64x32 pixel monochrome display and handling user input.}
%           \resumeItemListEnd

\resumeSubHeadingListEnd

%-----------PROGRAMMING SKILLS-----------
\section{Technical Skills}
\begin{itemize}[leftmargin=0.15in, label={}]
  \small{\item{
        \textbf{Languages}{: Python, Java, JavaScript, SQL (SQLite \& MySQL), C, HTML/CSS, Rust, OCaml} \\
        \textbf{Frameworks}{: Flask, Bootstrap} \\
        \textbf{Libraries}{: pandas, NumPy, python-pptx, jQuery, DataTables} \\
        % \textbf{Operating Systems}{: Windows, Linux, MacOS} \\
        \textbf{Developer Tools}{: Git, VS Code, Visual Studio, PyCharm, IntelliJ, Eclipse} \\
        }}
\end{itemize}


%-------------------------------------------
\end{document}
